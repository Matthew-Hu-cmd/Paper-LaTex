%!TEX program=xelatex

% 碰到Windows版本提示Fandol字体,可以在命令行中以管理员权限执行:tlmgr update -self -all
%\documentclass[review]{cvpr}
\documentclass[final]{cvpr}

\usepackage[UTF8]{ctex}

%\usepackage{cvpr}
\usepackage{times}
\usepackage{epsfig}
\usepackage{graphicx}
\usepackage{amsmath}
\usepackage{amssymb}
\usepackage{subfigure}
\usepackage{overpic}
% \usepackage{float}
\usepackage{stfloats}
\usepackage{enumitem}
\usepackage[noend]{algpseudocode}   %
\usepackage{setspace}   %
\usepackage{multirow}   %
\usepackage{diagbox}    %
\usepackage{booktabs}   %
\setenumerate[1]{itemsep=0pt,partopsep=0pt,parsep=\parskip,topsep=5pt}
\setitemize[1]{itemsep=0pt,partopsep=0pt,parsep=\parskip,topsep=5pt}
\setdescription{itemsep=0pt,partopsep=0pt,parsep=\parskip,topsep=5pt}


\usepackage[pagebackref=true,breaklinks=true,colorlinks,bookmarks=false]{hyperref}


%\cvprfinalcopy % *** Uncomment this line for the final submission

\def\cvprPaperID{159} % *** Enter the CVPR Paper ID here
\def\confYear{CVPR 2020}
\def\httilde{\mbox{\tt\raisebox{-.5ex}{\symbol{126}}}}

\newcommand{\cmm}[1]{\textcolor[rgb]{0,0.6,0}{CMM: #1}}
\newcommand{\todo}[1]{{\textcolor{red}{\bf [#1]}}}
\newcommand{\alert}[1]{\textcolor[rgb]{.6,0,0}{#1}}

\newcommand{\IT}{IT\cite{98pami/Itti}}
\newcommand{\MZ}{MZ\cite{03ACMMM/Ma_Contrast-based}}
\newcommand{\GB}{GB\cite{conf/nips/HarelKP06}}
\newcommand{\SR}{SR\cite{07cvpr/hou_SpectralResidual}}
\newcommand{\FT}{FT\cite{09cvpr/Achanta_FTSaliency}}
\newcommand{\CA}{CA\cite{10cvpr/goferman_context}}
\newcommand{\LC}{LC\cite{06acmmm/ZhaiS_spatiotemporal}}
\newcommand{\AC}{AC\cite{08cvs/achanta_salient}}
\newcommand{\HC}{HC-maps }
\newcommand{\RC}{RC-maps }
\newcommand{\Lab}{$L^*a^*b^*$}
\newcommand{\mypara}[1]{\paragraph{#1.}}

\graphicspath{{figures/}}

% Pages are numbered in submission mode, and unnumbered in camera-ready
%\ifcvprfinal\pagestyle{empty}\fi
\setcounter{page}{1}


\renewcommand{\figref}[1]{图\ref{#1}}
\renewcommand{\tabref}[1]{表\ref{#1}}
\renewcommand{\equref}[1]{式\ref{#1}}
\renewcommand{\secref}[1]{第\ref{#1}节}
\def\abstract{\centerline{\large\bf 摘要} \hspace*{12pt} \it}

%%%%%%%%% TITLE 

\title{伪装物体检测}

% \maketitle 

\author{Deng-Ping Fan$^{1,2}$\quad Ge-Peng Ji$^{3}$ \quad Guolei Sun$^{4}$
    \quad Jianbing Shen$^{1,}$\thanks{通讯作者:Jianbing Shen (shenjianbingcg@gmail.com)}  \quad Ling Shao$^{1}$  \\
    $^{1}$ Inception Institute of Artificial Intelligence, UAE \quad$^{2}$College of CS, Nankai University, China\\
    \quad$^{3}$ School of Computer Science,Wuhan University, China \quad $^4$ETH Zurich, Switzerland 
}

% \maketitle    %能用的位置

\begin{document}

\begin{figure*}[tp]%这个好像成了
\centering
\subfigure
{
    \begin{minipage}[htbp]{.18\linewidth}
        \centering
        \includegraphics[scale=0.25]{COD_Zh_translate/figures/example1.png}
        \hspace{5mm} %调整纵向距离
    \end{minipage}
}
\subfigure
{
 	\begin{minipage}[htbp]{.18\linewidth}
        \centering
        \includegraphics[scale=0.25]{COD_Zh_translate/figures/example2.png}
        \hspace{5mm} %调整纵向距离
    \end{minipage}
}
\subfigure
{
 	\begin{minipage}[htbp]{.18\linewidth}
        \centering
        \includegraphics[scale=0.25]{COD_Zh_translate/figures/example3.png}
        \hspace{5mm} %调整纵向距离
    \end{minipage}
}
\subfigure
{
 	\begin{minipage}[htbp]{.18\linewidth}
        \centering
        \includegraphics[scale=0.25]{COD_Zh_translate/figures/example4.png}
        \hspace{5mm} %调整纵向距离
    \end{minipage}
}
\subfigure
{
 	\begin{minipage}[htbp]{.18\linewidth}
        \centering
        \includegraphics[scale=0.25]{COD_Zh_translate/figures/example5.png}
        \hspace{5mm} %调整纵向距离
    \end{minipage}
}
\caption{COD10K 数据集示例图。你能找到隐藏在图片中的伪装物体吗?以彩色电子版阅览时视觉效果最佳。 答案见补充材料。}
\label{fig:COD10K_examples}
\end{figure*}


%%%%%%%%% ABSTRACT%%%%%%%%%%%
\begin{abstract}
伪装物体检测(Camouflaged Object Detection, COD),顾名思义,旨在识别“无缝”嵌入其周围环境 的物体,本文对这项新任务展开了全面的研究。与传 统的物体检测相比,通常伪装物体与其背景之间具 有高度相似性,因此伪装物体检测更具挑战。为解决 这一问题,本文精心构建了 COD10K 数据集,它包 含了 10,000 张图像,且涵盖了各种自然场景,具有超过78个类别的伪装物体。所有的图像都进行了稠密的标注,包括类别、包围盒、对象级/实例级,以及抠图级的标签。COD10K 数据集可以助力许多视觉任务,例如目标定位、图像分割和抠图技术等。同时, 本文也为伪装物体检测任务提供了一个简单且有效 的框架,称为搜索识别网络(Search Identification Network,SINet)。没有借助过多技巧,SINet 在所有 数据集上的表现均优于其它先进的物体检测基准模 型。因此,SINet 是一个鲁棒的、通用的架构,这有助 于促进伪装物体检测的发展。最后,通过对 13 种最 先进模型进行系统评估,本文给出了许多有趣的发 现并且展示了一些伪装物体检测的潜在应用。希望 本文的研究能为这一新领域的学者提供更多探索机 会。详见:https://github.com/DengPingFan/SINet/
\end{abstract}



%%%%%%%%% BODY TEXT %%%%%%%%%%%%%%%%%%%%%%%%%%%%%%%%%%%%%%%%
\section{引言}\label{sec:Introduction}
你能在\figref{fig:COD10K_examples}中找到伪装的物体吗?生物学家将
这类伪装方式称为背景匹配,即动物为避免被识别 [49],会尝试改变其自身颜色以“完美”地融入周围 环境。感官生态学家的研究表明,这种伪装策略是通 过欺骗观察者的视觉感知系统而产生的 [58]。因此, 解决伪装物体检测(Camouflaged Object Detection COD)任务需要大量的视觉感知知识 [61]。

\begin{figure}[h!]
  \begin{overpic}[width=\columnwidth]{Comparison_COD.png} \small
  \put(6,-4){(a)原图}
  \put(26,-4){(b)通用物体}
  \put(52,-4){(c)显著物体}
  \put(76,-4){(d)伪装物体}
    \end{overpic}\\
    \caption{给定一张输入图像(a),(b)为全景分割 [31] 的真值(全景分割检测中的通用物体 [40,45]包括stuff和things),(c)是显著实例/物体检测 [17, 34, 62, 77](检测最吸引人注意力的目标),(d)是本文提出的伪装物体检测任务,即检测出与 周围环境具有相似模式(例如边缘,纹理或颜色)的物体。如 图所示,两只蝴蝶的边缘与香蕉融合在一起,难以识别。
    }\label{fig:Comparsion_COD_SOD}
\end{figure}

如\figref{fig:Comparsion_COD_SOD}所 示,物体物与背景之间高度的相似性使 COD 远比 传统的显著物体检测(SOD)[1, 5, 18, 26, 63–67, 69] 或通用物体检测(GOD)[4,80] 更具挑战性。
除了其学术价值外,伪装物体检测还有助于推 动下列领域的实际应用:计算觉领域(可用于搜索 和救援工作,或寻找稀有物种)、医学图像分析领域
(如息肉分割 [15] 和肺炎分割 [19,68])、农业领域 (如蝗虫入侵监控)和艺术领域(用于真实感图像融
合 [22] 或艺术消遣 [6])。
目前,由于缺乏规模足够大的数据集,伪装物体检测的研究还不够深入。为了对 COD 课题进行
全面的研究,本文做出了两项
贡献。首先,本文专门 为COD任务精心构建了COD10K 数据集。它与 现有的数据集有以下方面的区别:
\begin{itemize}
    \item COD10K 数据集包含了1万张图像,涵盖了78 种伪装物体类别,属于水生、飞行、两栖和 陆地等。
    \item 所有的伪装图像都赋予了不同的层级标签,如 类别、包围盒、对象级和实例级。这些标签会使 得许多视觉任务受益,如目标定位,似物性检测,语义边缘检测 [43],任务迁移学习 [70] 等。
    \item 每张伪装图像都加上了真实环境中遇到的具有 挑战的属性以及抠图级 [74](标注一张图像耗时 约 60 分钟)的标签。这些高质量的标注有助于 对算法性能进行更深入的分析。
\end{itemize}%让文本空两行就能做到下一段、缩进


其次,使用本文构建的 COD10K 数据集和两个 现有数据集 [33,57],来共同构成最大的伪装物体检 测训练数据集,对 13 种最先进 (SOTA) 的基准模型 [3,24,28,33,36,39,41,52,69,76,78,79,83] 进行严格 的评估。本文的评估成为了目前最大规模的 COD 研 究。此外,本文还提供了一个简单而有效的框架,名 为 SINet (Search Identification Net)。值得注意的 是,SINet 的训练总时长仅为 1 小时左右,并且在目 前所有 COD 数据集上都达到了 SOTA 的性能。这 表明 SINet 可能是解决 COD 问题的潜在方案。在 深度学习时代下,本文是第一个完整的 COD 任务 评测,同时以伪装的角度去重新理解物体检测任务。





%%%%%%%%%%%%%%%%%%%%%%%%%%%%%%%%%%%%%%%%%%%%%%%%%%%%%%%%%%%%%%%%%%%%%%%%%%%%%%%%%
\begin{figure*}[tp]
    \centering
    \includegraphics[width=0.95\textwidth]{COD_Zh_translate/figures/COD10K_features.png}
    \caption{COD10K 数据集中各式各样的具有挑战性的属性示例图。属性表详见\tabref{tab:COD_Dataset}。以彩色电子版阅览视觉效果最佳。}
    \label{fig:COD10K_features}
\end{figure*}

\section{相关工作}
\label{sec:RelatedWorks}
根据 [80] 的研究,物体大致可以分为三类:通 用物体、显著物体和伪装物体。接下来本文将逐一介绍这些检测策略。
\subsection{通用物体和显著物体检测}
\mypara{通用物体检测(Generic Object Detection, COD)} 通用物体检测是计算机视觉中最热门的方向之一 [11,31,38,56]。值得注意的是,通用物体既可以 是显著的也可以是伪装的;伪装物体可视为通用物 体中较难的特例(如图. 9中第二行和第三行)。典型 的 GOD 任务包括语义分割和全景分割(见\figref{fig:Comparsion_COD_SOD} b)。
\mypara{显著物体检测(Salient Object Detection, SOD)}SOD 旨在识别图像中最引人注意的物体,然后对其 轮廓进行像素级分割 [29,39,73,78]。虽然“显著”一 词本质上与“伪装”相反(突出与浸入),但是显著 物体也可以为伪装物体检测提供重要信息,即可以 把包含显著对象的图像作为伪装物体检测的负样本。
%scalebox似乎是一种可行的压缩表格到目标大小的方法,但是似乎并不优雅,很多时候字体过小,用resizebox更加优雅
% \begin{table}
%     \scalebox{0.57}{
%     \centering
%     \begin{tabular}{c|ccc|ccccccc}
% \hline
% 数据集 & 年份 & 数量 & 类 & Att. & Bbox. & MI. & Ins. & Cate. & Spi. & Obj. \\ \hline
% CHAMELEON{[}57{]} & 2018 & 76   & -  &      &       &     &      &       &      & \checkmark    \\
% CAMO{[}33{]}      & 2019 & 2.5k & 8  & \checkmark    &       &     &      &       & \checkmark    & \checkmark    \\ \hline
% COD10K(Ours)      & 2020 & 10k  & 78 & \checkmark    & \checkmark     & \checkmark   & \checkmark    & \checkmark     & \checkmark    & \checkmark    \\ \hline
% \end{tabular}
%     }
%     \caption{从 COD 数据集对比信息来看,COD10K 数据集提 供了更丰富的注释标签。数量(Img.):图片数。类(Cls.): 类别。Att.:属性。BBox.;包围盒。Ml.:抠图 [74] 级标注
% (图. 7)。Ins.:实例级标签。Cate.:类别标签。Spi.:显式拆 分训练集与测试集。Obj.:物体。}\label{tab:COD_Dataset}
% \end{table}
\begin{table}
    \resizebox{\columnwidth}{!}{
    \centering
    \begin{tabular}{c|ccc|ccccccc}
\hline
数据集 & 年份 & 数量 & 类 & Att. & Bbox. & MI. & Ins. & Cate. & Spi. & Obj. \\ \hline
CHAMELEON{[}57{]} & 2018 & 76   & -  &      &       &     &      &       &      & \checkmark    \\
CAMO{[}33{]}      & 2019 & 2.5k & 8  & \checkmark    &       &     &      &       & \checkmark    & \checkmark    \\ \hline
COD10K(Ours)      & 2020 & 10k  & 78 & \checkmark    & \checkmark     & \checkmark   & \checkmark    & \checkmark     & \checkmark    & \checkmark    \\ \hline
\end{tabular}
    }
    \caption{从 COD 数据集对比信息来看,COD10K 数据集提 供了更丰富的注释标签。数量(Img.):图片数。类(Cls.): 类别。Att.:属性。BBox.;包围盒。Ml.:抠图 [74] 级标注
(图. 7)。Ins.:实例级标签。Cate.:类别标签。Spi.:显式拆 分训练集与测试集。Obj.:物体。}\label{tab:COD_Dataset}
\end{table}



\subsection{伪装物体检测}
伪装物体检测在生物学、艺术、医学等领域有着悠久而丰富的历史,对提高人类的视觉感知能力影 响巨大。Abbott Thayer[59]和 Hugh Cott[8]的关于伪装动物的两项杰出研究至今仍然影响广泛。感 兴趣的读者可以仔细阅读 Stevens等人[58]关于这段历史的描述。\mypara{数据集}CHAMELEON [57] 是一个未经同行评议的 数据集,仅包含 76 张图像,手工标注了对象级的真 值图(GTs)。这些图像是以“伪装的动物”为关键 字,通过谷歌搜索引擎收集的。另一个同期数据集 称为 CAMO [33],它具有 2500 张图像 (其中 2000 张用于训练,500 张用于测试),涵盖了八个类别。它 的两个子集 CAMO 和 MS-COCO,分别包含 1250 张图像。


不同于现有数据集,COD10K 数据集旨在提供一个更具挑战性,质量更高并且稠密标注的数据集。 COD10K 是迄今为止最大的伪装物体检测数据集, 包含有 1 万张图像(6000 张用于训练,4000 张用于 测试)。详见\tabref{tab:COD_Dataset}
\begin{figure*}
    \centering
    \includegraphics[width=0.95\textwidth]{COD_Zh_translate/figures/Catagory_COD10k.png}\small
    \caption{COD10K 数据集的统计信息和伪装类别示例。(a) 分类系统及其直方图分布。(b) 图像分辨率分布。(c) 词云分布。(d) 对象/实例的类别数量。(e) 子类示例。}
    \label{fig:Catagory_COD10k}
\end{figure*}
%这几段有大问题,很多数学公式里的符号都没有正常输入出来
\mypara{伪装的类型}伪装图像大致可分为两类,自然伪装和人工伪装的图像。动物 (例如昆虫、头足类动物) 的 自然伪装是一项生存技能,用来防止被捕食者识别。 相比之下,人工伪装通常被应用于视频监控环境;也 会出现在产品的生产过程中 (即产品缺陷检测);或 者在游戏或艺术中用来隐藏信息。
\mypara{COD定义} 和类别相关的语义分割任务不同,COD 任务与类别无关。因此,COD 任务简单且容易定义。 输入一张图像,通过伪装物体检测方法,为每个像 素i分配置信度pi ∈[0,1],其中pi 表示像素i的 概率值。0 表示此像素不属于伪装物体,而 1 则表 示完全属于伪装物体。本文主要研究对象级的 COD 任务,实例级 COD 作为未来的工作。
\mypara{评估指标}平均绝对误差(Mean absolute error, MAE)已被广泛用于 SOD 任务。和 Perazzi 等 人 [50] 一样,本文也使用了平均绝对误差 M 指 标(MAE(M))来评估预测图像 C 和真值图像G之间的像素级精度。尽管 MAE 指标能有效地评估错 误的出现和数量,但它不能评估错误出现的位置。最 近,范等人提出了一种基于人类视觉感知机制的增 强-匹配评价指标(E-measure,Eφ )[13],它同时考虑了像素级信息的匹配和图像级信息的统计。该评价 指标显然适用于评估伪装物体检测结果的整体和局 部精度。由于伪装物体通常包含复杂的形状,COD 还需要一个可以判断结构相似性的评价指标,因此 本文引入了 S 测评法(S-measure,Sα)[12]。最新 研究 [12,13] 表明,加权的 F 测评法(F-measure, Fβw )[44] 比传统的 Fβ 评测法更可靠。因此,本文 也在 COD 领域中使用这一评估指标。
%%%%%%%%%%%%%%%%%%%%%%%%%%%%%%%%%%%%%%%%%%%%%%%%%%%%%%%%%%
\section{文本数据集}\label{sec:Our_Dataset}
新任务和数据集 [7,11,37,48,82] 的出现促进了 计算机视觉各个领域的快速发展。例如,ImageNet [53] 数据集的出现使得深度模型在视觉识别任务上 成为了可能。本文构建 COD 数据集的初衷与之相 似,是为了:(1) 提供一项新的具有挑战性的任务, (2) 促进对新主题的研究,(3) 激发新想法。\figref{fig:COD10K_examples}、\figref{fig:COD10K_features}以及\figref{fig:Catagory_COD10k} (e)为 COD10K 的示例。接下来, 本文从以下三个方面详细阐述 COD10K 数据集。
\subsection{图像收集}
根据 [18,51] 的研究,数据集的标注质量和规 模决定了基准数据集的时间长短。为了延长其寿命, COD10K 数据集包含 10,000 张图像(5,066 张伪装 图像,3,000张背景图像,1,934张无伪装的图像),包含 10 个超类和 78 个子类(69 个伪装类,9 个非 伪装类),这些图像来源于多个摄影学网站。
大多数伪装图像来自 Flickr 网站,搜索关键字 为:camouflaged animal, unnoticeable animal, cam- ouflaged fish, camouflaged butterfly, hidden wolf spi- der, walking stick, dead-leaf mantis, bird, sea horse, cat, pygmy seahorses 等 (见\figref{fig:Catagory_COD10k} e),这些图像仅应 用于学术研究。其它伪装图像(约 200 张)来自下列 网站,如:Visualhunt,Pixabay,Unsplash,Free- images 等,这些图片不受版权和付费约束。为了避 免选择偏见 [18],我们还从 Flickr 收集了 3,000 张 显著图像。为了进一步丰富负样本,又从互联网上 下载了 1,934 张无伪装的图像,包括森林、雪地、草 地、天空、海水和其他类别的场景。有关图像选择方 案的更多信息,请见周等人的研究 [81]。

\begin{figure}[tp]
    \centering
    \includegraphics[width=0.98\columnwidth]{COD_Zh_translate/figures/Inter_relation_COD10k.png}\small
    \caption{COD10K 左图:COD10K 数据集的共生属性分布。网格中的数字为图像总数。右图:属性间的相互依赖关系。弧长越长则两属性间的相关性就越高。}
    \label{fig:inter_connection_COD10k}
\end{figure}

\begin{table}
    \resizebox{\columnwidth}{!}{
    \centering
    \begin{tabular}{cl}
        \hline
        属性 & 描述                                                                                          \\ \hline
        MO & 多物体 图像至少包含两个物体\\
        BO & 大物体 物体面积和图像面积的比值大于等于 0.5\\
        SO & 小物体 物体面积和图像面积的比值小于等于 0.1\\
        OV & 超出视野 物体的部分区域超出了图像边界\\
        OC & 遮挡 物体被部分遮挡。\\
        SC & 形状复杂 物体包含细小部分 (如:动物脚).\\
        IB & \begin{tabular}[c]{@{}l@{}}边缘不确定 目标周围区域的前景和背景具有相似的颜色 \\ (与RGB直方图χ2的距离τgc小于0.9)\end{tabular} \\ \hline
        \end{tabular}
            }
    \caption{属性描述 (示例见\figref{fig:COD10K_features})。}\label{tab:COD_Dataset}
\end{table}
\subsection{专业标注}
最新公布的数据集 [10, 16, 17] 表明,在创建大 规模数据集时,建立分类系统至关重要。受 [46] 启 发,本文进行了分层标注(采用众包方式标注)(类别$\longrightarrow$包围盒$\longrightarrow$属性$\longrightarrow$对象/实例)。
\begin{itemize}
    \item  \textbf{类别.}如\figref{fig:Catagory_COD10k}(a)所示,首先创建五个超类。 然后从收集到的数据中归纳了 69 个最常见的子类。 最后,标注每个图像的子类和超类。如果候选图像 不属于任何已有类别,则将其归为“其他”类。
    \item \textbf{包围盒.} 为了将COD10K数据集扩展到伪装物采样(proposal)任务,本文还细致地为每张图像 标注包围盒。
    \item \textbf{属性. }与文献 [18,51] 保持一致,本文采用自 然场景中常见的极具挑战的属性 (如遮挡、边缘不确 定),来标注每张图像。表. 2提供了属性描述,\figref{fig:inter_connection_COD10k}展示了共生属性分布情况。
\end{itemize}
%%%%%%%%%%%%%%%%%%%%%%%%%%%%%%%%%%%%%%%%%%%%%%%%%%%%%%%%%%%%%%%%这块的三个图是非常值得玩味的
%第一张图因为放在subsection3.3上面,不形成破坏,直接使用[tp]就上去了
%第2、3张图都是破坏的subsection3.3,所以可以放在这个subsection前面,然后统一自动排版
%如果最后一张图放置在subsection结束之后,很有可能又会单独成页
%%%%%%%%%%%%%%%%%%%%%%%%%%%%%%%%%%%%%%%%%%%%%%%%%%%%%%%%%%%%%%%
\begin{figure}[tp]
    \centering
    \includegraphics[width=0.98\columnwidth]{COD_Zh_translate/figures/Graph_analyse_COD10k.png}\small
    \caption{ COD10K 数据集与现有 COD 数据集相比,它包含小 物体(左上),包含更复杂的伪装(右上)并且受中心偏差影 响更小(左下/右下)。
}
    \label{fig:Graph_analyse_COD10k}
\end{figure}

\begin{figure}[bp]
    \centering
    \includegraphics[width=0.98\columnwidth]{COD_Zh_translate/figures/High_quality.png}\small
    \caption{ 高质量的抠图级标注 [74]。
}
    \label{fig:High_quality}
\end{figure}
%尤其是这张图的摆放位置,十分有参考价值
\begin{figure*}[t]
    \centering
    \includegraphics[width=0.98\textwidth]{COD_Zh_translate/figures/SINet_Struct.png}\small
    \caption{ SINet框架概要。本框架主要包括两个部分:感受野 (RF) 的感受野结构。PDC模块再现了动物捕食的搜索和识别阶段。SA和部分解码组件(PDC).RF模块用来模拟人类视觉系统中 为 [69] 中描述的搜索注意函数。详见 § 4。}
    \label{fig:SINet_Struct}
\end{figure*}

\subsection{数据集特点与统计信息}
\begin{itemize}
    \item \textbf{物体大小. }根据 [18] 的研究,在\figref{fig:Graph_analyse_COD10k}(左上 角)中绘制了归一化的物体尺寸,分布从 0.01\%到 80.74\%(平均:8.94\%)。与 CAMO-COCO、CPD1K 和 CHAMELEON 相比,本文数据集具有更广的尺寸范围。
    \item \textbf{全局/局部对比度.} 为了考察物体是否容易检 测,本文使用全局与局部比值评估方法 [35]。\figref{fig:Graph_analyse_COD10k}(右上角) 表明了 COD10K 数据集比其他数据集更具挑 战性。
    \item \textbf{中心偏见.} 由于人类会自然地将注意力集中在场景中心,因此摄影时就会产生这样的偏见。本 文采用 [18] 中的策略来分析这种偏见。\figref{fig:Graph_analyse_COD10k}(底部) 表明了本文数据集受到的中心偏见的影响更小。
    \item \textbf{质量控制. }为了确保高质量的标注,标记过程 邀请了三位观察者参与,并进行 10 组的交叉验证。 图. 7展示了通过和拒绝的标注示例。每张图像的实 例级标注平均耗时 60 分钟。
    \item \textbf{超类和子类的分布. }COD10K 包括 5 个超类 (陆地生物、空中生物、水生生物、两栖动物以及其 他类)和 69 个子类(例如:蝙蝠鱼, 狮子, 蝙蝠, 青 蛙等),各类别的词云和对象/实例数量的示例分别显示在\figref{fig:Catagory_COD10k} c 和 d 中。
    \item \textbf{分辨率分布. }[71] 研究表明,高分辨率图像可
以为模型训练提供更多物体边缘的细节从而在测试 时获得更好的性能。\figref{fig:Catagory_COD10k} (b)展示了 COD10K 数 据集图像分辨率的分布情况,它含有大量的 1080p 全高清图像。
    \item \textbf{数据集划分. }为了给深度学习模型提供大量 的训练数据,本文将 COD10K 数据集划分为:6,000 张图像的训练集和 4,000 张图像的测试集,图像分 别从每个子类中随机选择。
\end{itemize}

\section{本文框架}
\mypara{动机}生物学的研究 [23] 表明,捕食者在狩猎时,首 先会判断是否存在潜在猎物,即搜索猎物;然后,目 标动物被识别;最后,动物被捕获。
\mypara{概述}本文的 SINet 框架受到狩猎过程的前两阶段 的启发。框架主要包括两个模块:搜索模块 (Search Module, SM) 和识别模块 (Identification Module, IM)。前者 (\secref{sec:search_module}) 负责搜索被伪装的物体,而后者 (\secref{sec:identification_module}) 则用于精确检测物体。


\subsection{搜索模块(Search Module,SM)}\label{sec:search_module}
神经科学实验已证实,在人类视觉系统中,一组多尺度的群感受野(population Receptive Fields,pRFs)有助于让靠近视网膜中央凹的区域更加显著,而该区域对小的空间位移非常敏感[42]。于是,本文在搜索阶段(通常是在较小的、局部空间中)使用 RF [42,69] 模块来整合更具鉴别性的特征表5示。具 体而言,对于输入图像 
$\mathbf{I} \in \mathbb{R}^{W\times H\times3 }$ %%%行内公式的插入方式%%
,可利用 ResNet-50[25] 模型提取出一组特征
$\left \{ \chi_{k}  \right \}_{k=0}^{4}$。
为了保留更多信息,本文将第二层特征层的步长参数设置为1,使其和输入图像具有相同的分辨率。因此,每一层分辨率为
$\left \{ \left [  \frac{H}{k},\frac{W}{k}  \right ] , k=4,4,8,16,32\right \} $
最新研究 [79] 显示,更浅的卷积层中的低级特征保留了用于构建物体边缘的空间信息,而深层的深层卷积层的特征保留了用于定位目标的语义信息。由于神经网络本身的固有的特性,本文将提取的特征进行分层:低层 
$\left \{ \chi_{0},\chi_{1} \right \} $,中层 $\chi_{2}$ 和高层
$\left \{ \chi_{3},\chi_{4} \right \} $,并通过拼接、上采样和下采样等操作进行组合。与 [79] 不同,本文的 SINet 采用稠密连接策略 [27] 来保存来自不同特征层的更多信息,然后使用改进的RF[42]组件来扩大感受野。例如,先使用拼接操作来融合低级特征 
$\left \{ \chi_{0},\chi_{1} \right \} $,然后将分辨率下采样为原始一半。再将融合后的新特征
$rf_{4}^{x1x2} $进一步输入到RF组件生成$rf_{4}^{s}$特征。如\figref{fig:SINet_Struct}所示,在组合了三个层次的特征之后,得到了一组增强的特征$\left\{rf_{k}^{s},k=1,2,3,4\right\}$,用于鲁棒地学习伪装线索。\textbf{感受野(Receptive Field, RF). }RF模块包括五个分支$\left \{ b_{k} , k=1,...,5 \right \}$。在每个分支中,第一个 卷积层 (Bconv) 的尺寸为$1\times1$,用以将通道数降为32。其后两层分别为:$(2k-1)\times(2k-1)$Bconv层和3×3Bconv层。当k>2时,空洞卷积率设置为$(2k − 1)$。前四个分支串联后,通过 1 × 1 Bconv 操 作,其通道数降为 32。最后,加入第 5 个分支,并 整体输入进 ReLU 函数以获得特征$rf_{k}$。


\subsection{识别模块(Identification Module, IM)}\label{sec:identification_module}
通过之前的搜索模块获取到侯选特征后,在识 别模块中,需要对伪装物体进行精确检测。本文采用密集连接方式对部分解码组件 (Partial Decoder Component,PDC) [69] 进行了扩展。具体来讲,PDC 整合了来自 SM 的四个特征层。可通过以下公式来 计算粗糙的伪装图$C_{s}$:
\begin{equation}\label{equ:Partial_Decoder_Component}
   C_{s} =PD_{s} \left ( rf_{1}^{s}, rf_{2}^{s}, rf_{3}^{s}, rf_{4}^{s}\right),
\end{equation}
其中$\left\{rf_{k}^{s}=r_{k},k=1,2,3,4\right\}$。现有文献[41,69]已表明,注意力机制可以有效地消除无关特征的干扰。因此引入搜索注意力 (Search Attention,SA) 模块 来增强中间特征层$\chi_{2}$并获得增强的伪装图$C_{h}$:
\begin{equation}\label{equ:Search_Attention}
   C_{h} = f_{max}\left ( g(\chi_{2}, \sigma, \lambda), C_{s} \right ),
\end{equation}
其中$g\left ( \cdot \right ) $是SA函数,即为典型的归一化后的高斯滤波器,其标准差为:$\sigma = 32 $核尺寸为:$ \lambda = 4$,$f_{max}\left ( \cdot \right )$是一个最大化函数,用来突出伪装图$C_{s}$初始的伪装区域。


为了全面获取高层特征,本文进一步使用 PDC 来聚合另外三层的特征,并通过 RF 进行增强,以获得最终的伪装图$C_{i}$:
\begin{equation}\label{equ:RF_Partial_Decoder_Component}
   C_{i} = PD_{i} \left ( rf_{1}^{i}, rf_{2}^{i}, rf_{3}^{i} \right),
\end{equation}
其中$\left \{ rf_{k}^{i} = rf_{k},k=1,2,3 \right \}$。$PD_{s}$和$PD_{i}$之间的
差别是输入特征的数量不同。
\mypara{部分解码组件(Partial Decoder Component,PDC)}
形式上,给定一组来自搜索和识别阶段获取的特征$\left \{ rf_{k}^{c} , k\in \left [ m,...,M \right ]   ,C \in \left [ s,i \right ] \right \} $本文使用上下文模块来生成新特征$\left \{ rf_{k}^{c1} \right \} $。并采用逐元素相
乘方式来减少相邻特征之间的差距。具体来说,对于最浅层的特征,如$rf_{4}^{s}$,当$k=M$时,令$rf_{M}^{c1}=rf_{M}^{c2}$,当$K<M$时,则将$rf_{k}^{c2}$更新为:
\begin{equation}\label{equ:PDC_refresh}
   rf_{k}^{c2} =rf_{k}^{c1} \otimes  {\textstyle \prod_{j=k+1}^{M}}Bconv\left ( UP\left ( f_{j}^{c1}  \right )  \right )  ,
\end{equation}
其中$k\in \left [ m,...,M−1 \right ],Bconv(\cdot) $ 是结合了 3×3 卷 积、批量归一化和 ReLU 函数的顺序操作。$UP(\cdot) $为具有$2j−k$倍的上采样操作。最后,通过拼接方式将 这些具有区分性的特征组合在一起。训练$SINet$模 型损失函数为交叉熵[78]损失函数$L_{CE}$ 。总损失函数$L$为:
\begin{equation}\label{equ:L_CE_loss}
   L=L_{CE}^{s}\left ( C_{csm},G \right )  +L_{CE}^{i} \left ( C_{cim},G \right ) ,
\end{equation}
其中$C_{csm}$和$C_{cim}$是对$C_{s}$和$C_{i}$上采样后获得的两个伪装物体映射图,其分辨率为:352×352。


\subsection{实施细节}
$SINet$是在 PyTorch中实现的,并使用Adam优化器进行训练[30]。在训练阶段,批处理大小为 36,学习率从1e-4开始。整个训练过程共有30个阶段(采用提前停止策略),仅耗时大约70分钟。运行时间是在以下平台上测试获得:Intel® i9-9820X CPU @3.30GHz × 20、TITAN RTX。分辨率为 352×352 的图像,其推理时间为 0.2s。
%%%%%%%%%%%%%%%%%%%%%%%%%%%%%%%%%%%%%%%%%%%%%%%%%%%%%%%%%%%%%%%%%%%%%%
%大的表格自适应调整这么弄就对了
%%%%%%%%%%%%%%%%%%%%%%%%%%%%%%%%%%%%%%%%%%%%%%%%%%%%%%%%%%%%%%%%%%%%%%
\begin{table*}[pb]
\resizebox{\textwidth}{!}
    {
        \begin{tabular}{l|llll||llll||llll}
        \hline
        \multicolumn{1}{c|}{\multirow{2}{*}{基准模型}} &
          \multicolumn{4}{c||}{CHAMELEON {[}57{]}} &
          \multicolumn{4}{c||}{CAMO-Test {[}33{]}} &
          \multicolumn{4}{c}{COD10K-Test   (Ours)} \\ \cline{2-13} 
        \multicolumn{1}{c|}{} &
          \multicolumn{1}{c}{$S_{\alpha } \uparrow$ } &
          \multicolumn{1}{c}{$E_{\phi } \uparrow$ } &
          \multicolumn{1}{c}{$F_{\beta }^{\omega } \uparrow $} &
          \multicolumn{1}{c||}{$M\downarrow $} &
          \multicolumn{1}{c}{$S_{\alpha } \uparrow$} &
          \multicolumn{1}{c}{$E_{\phi } \uparrow$} &
          \multicolumn{1}{c}{$F_{\beta }^{\omega } \uparrow $} &
          \multicolumn{1}{c||}{$M\downarrow $} &
          \multicolumn{1}{c}{$S_{\alpha } \uparrow$} &
          \multicolumn{1}{c}{$E_{\phi } \uparrow$} &
          \multicolumn{1}{c}{$F_{\beta }^{\omega } \uparrow $} &
          \multicolumn{1}{c}{$M\downarrow $} \\ \hline
        2017 FPN {[}36{]}               & 0.794 & 0.783 & 0.591 & 0.075 & 0.794 & 0.783 & 0.591 & 0.075 & 0.794 & 0.783 & 0.591 & 0.075 \\
        2017 MaskRCNN {[}24{]}          &       &       &       &       &       &       &       &       &       &       &       &       \\
        2017 PSPNet {[}76{]}            &       &       &       &       &       &       &       &       &       &       &       &       \\
        2018 UNet++ {[}83{]}            &       &       &       &       &       &       &       &       &       &       &       &       \\
        2018 PiCANet {[}41{]}           &       &       &       &       &       &       &       &       &       &       &       &       \\
        2019 MSRCNN {[}28{]}            &       &       &       &       &       &       &       &       &       &       &       &       \\
        2019 BASNet {[}52{]}            &       &       &       &       &       &       &       &       &       &       &       &       \\
        2019 PFANet {[}79{]}            &       &       &       &       &       &       &       &       &       &       &       &       \\
        2019 CPD {[}69{]}               &       &       &       &       &       &       &       &       &       &       &       &       \\
        2019 HTC {[}3{]}                &       &       &       &       &       &       &       &       &       &       &       &       \\
        2019 EGNet {[}78{]}             &       &       &       &       &       &       &       &       &       &       &       &       \\
        2019 ANet-SRM {[}33{]}          &       &       &       &       &       &       &       &       &       &       &       &       \\ \hline
        SINet’20 Training setting (i)   &       &       &       &       &       &       &       &       &       &       &       &       \\
        SINet’20 Training setting (ii)  &       &       &       &       &       &       &       &       &       &       &       &       \\
        SINet’20 Training setting (iii) &       &       &       &       &       &       &       &       &       &       &       &       \\ \hline
        \end{tabular}
    }
    \caption{各数据集上的定量评估结果。最高分以粗体突出。训练设置见\secref{sec:Experiment_settings}:(i) CAMO, (ii) COD10K, (iii) CAMO + COD10K + EXTRA. 注意,ANet-SRM模型码 (仅在 CAMO 上训练) 没有开源代,因此,因此其他数据集的结果无法得知(‘‡’)。$\uparrow$表示评分越高越好。$E_{\phi }$表示E测评法[13]的均值. 基准模型使用第 (iv) 集合进行训练。评测代码详见: https://github.com/DengPingFan/CODToolbox}
    \label{tab:results}
\end{table*}
\section{评测实验}\label{sec:Experiment}
\subsection{实验设置}\label{sec:Experiment_settings}
\mypara{训练和测试细节}
为了验证$SINet$的通用性,使用以下三种训练集(伪装图像):\textbf{(i)}CAMO [33];\textbf{(ii)} COD10K;\textbf{(iii)}CAMO + COD10K + EXTRA。对于 CAMO,则使用默认的训练集。同样对于COD10K,也使用默认的伪装图像训练集。本文在以下数据集 上进行模型评估:整个 CHAMELEON [57] 数据集,CAMO和COD10K 两个数据集对应的测试集。\mypara{基线}由于没有公开的基于深度网络的 COD 模型。因此,本文根据以下标准选择了 12 个深度学习的基 准模型 [3, 24, 28, 33, 36, 41, 52, 69, 76, 78, 79, 83] 并按 照第 iv 个训练方案,使用原文推荐的参数来训练这 些基准模型。(1) 经典框架,(2) 最新发布,(3) 在特 定领域 (例如 GOD 或 SOD) 中达到 SOTA 性能。
\begin{table}[pt]
    \resizebox{\columnwidth}{!}
    {
    \centering
    \begin{tabular}{r|ll|ll|l} 
    \toprule
    \multicolumn{1}{l|}{\diagbox{训练:}{测试:}} & \multicolumn{1}{c}{\begin{tabular}[c]{@{}c@{}}CAMO\\{[}33]\end{tabular}} & \multicolumn{1}{c|}{\begin{tabular}[c]{@{}c@{}}COD10K\\(Ours)\end{tabular}} & \multicolumn{1}{c}{~ 自身~~} & \multicolumn{1}{c|}{\begin{tabular}[c]{@{}c@{}}~ 其他~~\\~ 均值~~\end{tabular}} & \multicolumn{1}{c}{~ 下降$\downarrow$}  \\ 
    \hline
    CAMO [33]   & \textbf{0.803}    & 0.702         & 0.803 & 0.678 & 15.6\%    \\
    COD10K (Ours)  & 0.742          & 0.7\textbf{00}& 0.700 & 0.683 & 2.40\%    \\ \hline
    其他均值    & 0.641             & 0.589         &       &       &           \\
    \bottomrule
    \end{tabular}
    }
    \caption{$SINet$模型在跨数据集下得到的 S 测评法 (S-measure$\uparrow$[12]) 结果。$SINet$ 在某一行上的对应数据集上进 行训练,然后分别在列上的所有数据集上进行测试。“自身 (Self)”:表示训练和测试都在同一个(对角线上)数据集上 进行。“其他均值 (Mean others)”表示在其他数据集上的平 均分数。}\label{tab:S_measure_result}
\end{table}
\begin{figure*}
    \centering
    \includegraphics[width=\textwidth]{COD_Zh_translate/figures/Comparesion_SINet_EGNet_PFANet.png}\small
    \caption{SINet 和两个在COD10K数据集上表现最好的基准模型的定性分析。细节详见补充材料。}
    \label{fig:Comparesion_SINet_EGNet_PFANet}
\end{figure*}

\subsection{结果与数据分析}
\mypara{CHAMELEON 性能表现}从\tabref{tab:results}中可以看出,相比12个SOTA的物体检测基准,本文提出的 SINet 在所有指标上均获胜。尤其是本文模型并未使用任何边缘/边界特征(如EGNet[78],PFANet [79] )、预处理技术 [47] 或后处理策略 (如 CRF [32],图 割模型 [2])。
\mypara{CAMO 性能表现}本文还在最近提出的CAMO[33]数据集上进行了模型测试,它包括了各种伪装物 体。根据\tabref{tab:results}中全面的分析,不难发现CAMO数据集比之前的两个数据集 (CHAMELEON,CPD1K) 更具挑战性。$SINet$再次获得最佳表现,因此进一步证明了它的鲁棒性。
\mypara{COD10K 性能表现}使用 COD10K 的测试集 (2,026 张图像) 进行测试,可以再次发现 $SINet$一如既往地优于其他模型。这是因为其专门设计的搜索和识别模块可以自动学到丰富的高、中、低层的 特征,而这些正是解决伪装物体边缘难以确定的这 一具有挑战性问题的关键 (见\figref{fig:Comparesion_SINet_EGNet_PFANet})。
\mypara{GOD 与 SOD 的基准模型}值得注意,在排名 前三的模型中,GOD 模型 (FPN [36]) 比 SOD的CPD[69]和EG-Net[78]表现差,这说明SOD框架可能更适合扩展到COD任务上。不论与 GOD [3, 24, 28, 36, 76, 83] 模型还是 SOD [39, 41, 52, 69, 78, 79]
模型相比,$SINet$明显缩短了训练时间 (例如,$SINet$: 耗 1小时而$EGNet$需要耗费48 小时),并且在所有数据集上都达到了SOTA性能,这表明本模型有希望解决COD问题。
\mypara{数据集间的泛化性}数据集的泛化性和难度在训练 和评估不同算法 [62] 中都起着至关重要的作用。因 此,本文使用跨数据集分析法 [60],对现有 COD 数 据集进行研究,即在一个数据集上训练模型,然后在其他数据集上进行测试。本文选择了两个数据集,包括 CAMO[33]和本文构建的$COD10K$数据集。参考[62],本文也对于每个数据集,随机选择 800张图像作为训练集,选择200张图像作为测试集,因为$CPD1K$数据集仅包含 1,000 张图像。为公平比较,$SINet$模型在每个数据集上都训练到损失稳定为止。


\tabref{tab:S_measure_result}提供了在跨数据集上使用 S 测评法 (S-
measure) 的结果。每行数据表示用这一行的数据集训练,再对各个列中的数据集进行测试。因此,从行的角度看,可以得知该行数据集的泛化性。每列数据 表示在对应行数据集上进行训练后,再对列上数据集上进行测试,因此,从列的角度分析,可知该列数据集的难度。请注意,训练和测试的设置与表. 3中 使用的设置不同。因此性能无法相互比较。正如预期,本文的 COD10K 是难度最大的数据集(即,最后一行的其他平均为:0.589)。这是因为数据集包含了各种具有挑战性的伪装物体(见\secref{sec:Our_Dataset})。
\mypara{定性分析}\figref{fig:Comparesion_SINet_EGNet_PFANet}给出了本文$SINet$模型与两个基准模型之间的定性比较。观察发现,PFANet[79]虽然 定位到伪装物体,但是输出总是不准确。通过进一步使用边缘特征,EGNet[78]获得了比PFANet相对更准确的定位。然而,它仍然忽略了物体的细微之处,尤其是第一行的鱼类。所有这些具有挑战的示例(例如,边缘不确定,遮挡和小物体),$SINet$模型均可检测出伪装物体的精确细节从而预测出真正 伪装物体,展示了本文框架的稳健性。
%%%%%%%%%%%%%%%%%%%%%%%%%%%%%%%%%%%%%%%%%%%%%%%%%

\begin{figure}[tp]
    \centering
    \includegraphics[width=\columnwidth]{COD_Zh_translate/figures/More_Application.png}\small
    \caption{更多应用示例。(a) 息肉检测/分割结果。(b) 用于灾 区的搜救系统。}
    \label{fig:More_Application}
\end{figure}

\section{潜在应用}\label{sec:Application_to_be_found}
伪装检测系统(Camouflage Detection Systems,CDS)
有许多潜在的应用。本文在此预想了两种潜在用途。
\mypara{医学图像分割}配备了CDS的医疗图像分割算法,用息肉等特殊对象训练后,则可以用于在真实场景中自动分割息肉
(\figref{fig:More_Application} a),
或者在自然界中发现和保护稀有物种,甚至在灾难场景中用于搜索和救援。 
\mypara{搜索引擎}\figref{fig:Application_SearchEngin}是一个谷歌的搜索结果示例。从结果(\figref{fig:Application_SearchEngin} a) 中,可以注意到搜索引擎无法检测到隐藏的蝴蝶,因此只能搜索到具有相似背景的图像。有趣的是,当搜索引擎配备了CDS(在此,只是简单地修改了关键字),引擎就可以识别伪装物体,并且反馈一些蝴蝶的图像
(\figref{fig:Application_SearchEngin} b)。
\\
\\
\\
\\
\\
\\
\\
\\
\\
\\
\\
\\
\\

\section{结论}\label{sec:Conclusion}
本文从伪装角度为物体检测构建了一个完整的评测。具体来说,本文构建了一个新的、有挑战性的、稠密标注的COD10K数据集,并进行了大规模的评估,开发了一个简单而有效的端到端$SINet$框架,并展示了一些潜在应用。与目前先进基准模型相比,$SINet$框架极具竞争力,并且在视觉上能够得到更好的结果。以上贡献旨在为本领域提供一个为COD任务设计新模型的机会。在未来的工作中,我们计划扩展COD10K数据集,可提供各种形式的输入,例如RGB-D伪装物体检测(类似于RGB-D显着物体检测[20,72,75]),或者其他形式。探索一些新技术,例如弱监督学习[54,55]、零样本学习[84]、变分自动编码模型[85]和多尺度骨干网络 [21]。

\begin{figure}[tp]
    \centering
    \includegraphics[width=\columnwidth]{COD_Zh_translate/figures/Application_SearchEngin.png}\small
    \caption{配备 CDS 与否的搜索引擎示例,(a):无 CDS,(b)
    有 CDS。}
    \label{fig:Application_SearchEngin}
\end{figure}

\begin{figure*}[tp]
    \centering
    \includegraphics[width=\textwidth]{COD_Zh_translate/figures/ref1.png}\small
\end{figure*}
\begin{figure*}[tp]
    \centering
    \includegraphics[width=\textwidth]{COD_Zh_translate/figures/ref2.png}\small
\end{figure*}
\begin{figure*}[tp]
    \centering
    \includegraphics[width=\textwidth]{COD_Zh_translate/figures/ref3.png}\small
\end{figure*}
%引文我就只能意思一下了,COD原文引用太多了,我这边随便找齐了百多篇放到了bib里面,然后到时候我只能说随便引了,最后用goodnotes弄一下pdf

% {\small
% \bibliographystyle{ieee}
% \bibliography{COD_fake}
% }

% \end{CJK*}
\end{document}
